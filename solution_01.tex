\documentclass[a4paper,12pt,headsepline]{scrartcl}

%\part{title}
\usepackage[utf8]{inputenc}
\usepackage{graphicx}
\usepackage{caption,subcaption}
\usepackage[ngerman]{babel}
\usepackage[T1]{fontenc}
\usepackage{geometry}
\usepackage{proof}
\geometry{left=3.5cm, right=2cm, top=2.5cm, bottom=2cm}
\usepackage{hyperref}
%\usepackage[hyphens,obeyspaces,spaces]{url}
\usepackage{fancybox}
\usepackage{amssymb,amsmath,amsthm}
\usepackage{gensymb}
\usepackage[linesnumbered,ruled,vlined,norelsize]{algorithm2e}
%\usepackage[bookmarksnumbered,pdftitle={\titleDocument},hyperfootnotes=false]{hyperref} 
\usepackage{color}
\usepackage{float}

%test
%\usepackage[backend=bibtex]{biblatex}
%\usepackage{filecontents}

%\addbibresource{ref.bib}

\restylefloat{figure}

% Makros
%\newenvironment{sketch}{\begin{proof}[Proof (Sketch)]}{\end{proof}}
%\newtheorem{theorem}{Theorem}
%\newtheorem{assumption}{Assumption}
%\newtheorem{lemma}{Lemma}
%\newtheorem{remark}{Remark}
%\newtheorem{definition}{Definition}
%\newtheorem{corollary}{Corollary}
\newcommand{\comment}[1]
{
  \begin{quotation}
    \textcolor{blue}{\underline{Edit:} #1}
  \end{quotation}
}
\newtheorem{aufgabe}{Aufgabe}
\newcommand{\Ex}[2]
{
	\setcounter{section}{#2}
	\section*{Übungsblatt #2 zu #1}
}
\newcommand{\TODO}[1]
{
  \begin{quotation}
    \textcolor{red}{\underline{TODO:} #1}
  \end{quotation}
}
% Zeichen 
\newcommand{\OO}{\ensuremath{\mathcal{O}}}
\newcommand{\ec}{\texttt{ec}}
\newcommand{\NP}{\call{NP}}
\newcommand{\call}[1]{\ensuremath{\mathcal{#1}}}

% neue Kopfzeilen mit fancypaket
\usepackage{fancyhdr} %Paket laden
\pagestyle{fancy} %eigener Seitenstil
\fancyhf{} %alle Kopf- und Fußzeilenfelder bereinigen
\fancyhead[L]{Benjamin \c Coban \\ Michael Throner \\ Philipp Huber }
\fancyhead[C]{Parallele Rechnerarchitekturen \\ Blatt 1}
\fancyhead[R]{3526251 \\ 4019521 \\ 3776454}
\setlength{\headheight}{39pt}
\renewcommand{\headrulewidth}{0.4pt} %obere Trennlinie
%\fancyfoot[C]{\thepage} %Seitennummer
%\renewcommand{\footrulewidth}{0.4pt} %untere Trennlinie

\frenchspacing
\makeindex

% Pseudocode für Java
\usepackage{listings}
\lstset{numbers=left, numberstyle=\tiny, numbersep=5pt, keywordstyle=\color{black}\bfseries, stringstyle=\ttfamily,showstringspaces=false,basicstyle=\footnotesize,captionpos=b}
\lstset{language=java}

% Disable single lines at the start of a paragraph (Schusterjungen)
\clubpenalty = 10000
% Disable single lines at the end of a paragraph (Hurenkinder)
\widowpenalty = 10000
\displaywidowpenalty = 10000

\begin{document}

%\begin{flushright}
%	Benjamin \c Coban ~~~~~ 3526251\\
%	Philipp Huber ~~~~~ 3776454\\
%	Michael Throner ~~~~~ 4019521
%\end{flushright}
%\Ex{Parallenen Rechnerarchitekturen}{1}
\section{Out-of-Order Execution}
\subsection*{a)}
$fdiv$ benötigt den initialen Wert von $s0$ muss somit vor $fmul$ ausgefürt werden.\\
$fadd[1]$ benötigt $s0$ von $fdiv$ und kann somit erst danach und vor $fmul$ berechnet werden. Außerdem wird der initiale Wert von $s2$ benötigt wodurch die Funktion auch vor $fsub$ berechnet werden muss.\\
$str$ braucht $s0$ von $fdiv$ und muss somit vor $fmul$ und nach $fdiv$ durchgeführt werden.\\
$fsub$ benötigt den Initialen Wert von $s4$ und muss somit vor $fadd[2]$ ausgeführt werden.\\
$fmul$ verwendet die Ergebnisse $s2$ von $fsub$ und $s4$ von $fadd[2]$. Somit kann $fmul$ erst nach beiden ausgeführt werden.
\\
\resizebox{\linewidth}{!}{
\begin{tabular}{l|c|c|c|c|c|c|c|c|c|c|c|c|c|c|c|c}
Befehl/Takt	&0	&1	&2	&...	&18	&19	&20	&21	&22	&23	&24	&25	&26	&27	&28	&29\\\hline\hline
$fdiv$		&	&RX	&RX	&...	&RX	&RXW	&	&	&	&	&	&	&	&	&	&\\\hline
$fadd[1]$	&	&	&	&...	&	&	&	&RX	&RXW	&	&	&	&	&	&	&\\\hline
$str$		&	&	&	&...	&	&	&RXW	&	&	&	&	&	&	&	&	&\\\hline
$fsub$	&	&	&	&...	&	&	&	&	&RX	&RXW	&	&	&	&	&	&\\\hline
$fadd[2]$	&	&	&	&...	&	&	&	&	&	&RX	&RXW	&	&	&	&	&\\\hline
$fmul$	&	&	&	&...	&	&	&	&	&	&	&	&RX	&RX	&RX	&RX	&RXW\\\hline\hline
Reg. \& Speicher/Takt	&0	&1	&2	&...	&18	&19	&20	&21	&22	&23	&24	&25	&26	&27	&28	&29\\\hline\hline
$s0$		&3,6	&3,6	&3,6	&...	&3,6	&1,8	&1,8	&1,8	&1,8	&1,8	&1,8	&1,8	&1,8	&1,8	&1,8	&19,2\\\hline
$s1$		&2,0	&2,0	&2,0	&...	&2,0	&2,0	&2,0	&2,0	&2,0	&2,0	&2,0	&2,0	&2,0	&2,0	&2,0	&2,0\\\hline
$s2$		&2,0	&2,0	&2,0	&...	&2,0	&2,0	&2,0	&2,0	&3,8	&4,0	&4,0	&4,0	&4,0	&4,0	&4,0	&4,0\\\hline
$s3$		&3,8	&3,8	&3,8	&...	&3,8	&3,8	&3,8	&3,8	&3,8	&3,8	&3,8	&3,8	&3,8	&3,8	&3,8	&3,8\\\hline
$s4$		&5,0	&5,0	&5,0	&...	&5,0	&5,0	&5,0	&5,0	&5,0	&5,0	&4,8	&4,8	&4,8	&4,8	&4,8	&4,8\\\hline
$s5$		&1,0	&1,0	&1,0	&...	&1,0	&1,0	&1,0	&1,0	&1,0	&1,0	&1,0	&1,0	&1,0	&1,0	&1,0	&1,0\\\hline
$MEM[0x10001000]$&0,0&0,0&0,0&...&0,0	&0,0	&1,8	&1,8	&1,8	&1,8	&1,8	&1,8	&1,8	&1,8	&1,8	&1,8\\\hline
\end{tabular}
}

\subsection*{b)}

$fdiv$	$s0,s0,s1$\\
$fadd$	$s2,s0,s2$\\
$str$	$s0,[r0,\#1000]$\\
$fsub$	$t0,s4,s5$\\
$fadd$	$s4,s3,s5$\\
$fmul$	$t1,t0,s4$\\

\resizebox{\linewidth}{!}{
\begin{tabular}{l|c|c|c|c|c|c|c|c|c|c|c|c|c|c|c|c|c}
Befehl/Takt	&0	&1	&2	&3	&4	&5	&6	&7	&8	&9	&10	&...	&18	&19	&20	&21	&22\\\hline\hline
$fdiv$		&	&RX	&RX	&RX	&RX	&RX	&RX	&RX	&RX	&RX	&RX	&...	&RX	&RXW	&	&	&\\\hline
$fadd[1]$	&	&	&	&	&	&	&	&	&	&	&	&...	&	&	&RX	&RXW	&\\\hline
$str$		&	&	&	&	&	&	&	&	&	&	&	&...	&	&	&	&	&RXW\\\hline
$fsub$	&	&	&RX	&RXW	&	&	&	&	&	&	&	&...	&	&	&	&	&\\\hline
$fadd[2]$	&	&	&	&RX	&RXW	&	&	&	&	&	&	&...	&	&	&	&	&\\\hline
$fmul$	&	&	&	&	&	&RX	&RX	&RX	&RX	&RXW	&	&...	&	&	&	&	&\\\hline\hline
Reg. \& Speicher/Takt	&0	&1	&2	&3	&4	&5	&6	&7	&8	&9	&10	&...	&18	&19	&20	&21	&22\\\hline\hline
$s0$		&3,6	&3,6	&3,6	&3,6	&3,6	&3,6	&3,6	&3,6	&3,6	&3,6	&3,6	&...	&3,6	&1,8	&1,8	&1,8	&1,8\\\hline
$s1$		&2,0	&2,0	&2,0	&2,0	&2,0	&2,0	&2,0	&2,0	&2,0	&2,0	&2,0	&...	&2,0	&2,0	&2,0	&2,0	&2,0\\\hline
$s2$		&2,0	&2,0	&2,0	&2,0	&2,0	&2,0	&2,0	&2,0	&2,0	&2,0	&2,0	&...	&2,0	&2,0	&2,0	&3,8	&3,8\\\hline
$s3$		&3,8	&3,8	&3,8	&3,8	&3,8	&3,8	&3,8	&3,8	&3,8	&3,8	&3,8	&...	&3,8	&3,8	&3,8	&3,8	&3,8\\\hline
$s4$		&5,0	&5,0	&5,0	&5,0	&4,8	&4,8	&4,8	&4,8	&4,8	&4,8	&4,8	&...	&4,8	&4,8	&4,8	&4,8	&4,8\\\hline
$s5$		&1,0	&1,0	&1,0	&1,0	&1,0	&1,0	&1,0	&1,0	&1,0	&1,0	&1,0	&...	&1,0	&1,0	&1,0	&1,0	&1,0\\\hline
$t0$		&1,0	&1,0	&1,0	&1,0	&4,0	&4,0	&4,0	&4,0	&4,0	&4,0	&4,0	&...	&4,0	&4,0	&4,0	&4,0	&4,0\\\hline
$t1$		&1,0	&1,0	&1,0	&1,0	&1,0	&1,0	&1,0	&1,0	&1,0	&19,2	&19,2	&...	&19,2	&19,2	&19,2	&19,2	&19,2\\\hline
$MEM[0x10001000]$&0,0&0,0&0,0&0,0&0,0	&0,0	&0,0	&0,0	&0,0	&0,0	&0,0	&...	&0,0	&0,0	&0,0	&0,0	&1,8\\\hline
\end{tabular}
}

\section{Out-of-Order Execution mit gepipelineten Ausführungseinheiten}

In einer voll gepipelineten Ausführungseinheit kann pro Takt die Ausführung einer Instruktion begonnen werden. Eine solche Einheit kann also die selbe Ausführungsordnung gewährleisten, die in Aufgabe 1 vorgeschlagen wurde und hält dabei alle Bedingungen ein. Diese Ausführungsordnung gewährleistet auch dass keine WAR-Hazards auftreten können, da Instruktionen die Registerinhalte überschreiben nicht gestartet werden bevor der Registerinhalt von allen notwendigen Instruktionen gelesen wurde. 

\end{document}